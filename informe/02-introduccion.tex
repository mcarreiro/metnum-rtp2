\section{Introducci\'on te\'orica}

\subsection{Algoritmos de búsqueda}

Los buscadores web son aquellos sitios que dada una palabra o frase encuentran las páginas que tienen lo que el usuario estaba buscando en una gran cantidad de casos. El secreto del éxito de estos buscadores no es sólo que encuentran el sitio con la palabra o frase deseada sino que encuentran el mejor resultado para esta. \\
Esto lo hace mediante métodos de rankeo que no son nada sencillos y es por eso que estaremos analizando 3 algoritmos distintos para hacerlo: PageRank, HITS e Indeg. Describiremos la idea teórica de cada uno, sus implementaciones y analizaremos cualitativa y temporalmente sus resultados. Es decir cuanto tardan en ejecutarse y que tan buenos son los resultados devueltos para distintas búsquedas. Procedamos entonces a explicar la idea de cada uno.

\subsubsection{PageRank}

El algoritmo de rankeo de sitios  llamado \textbf{PageRank} lo desarrollaron los fundadores de Google, Page y Brin, y fue tan revolucionario que cambió la forma en que se empezó a buscar en la web. \\
El primer punto de este algoritmo es mirar a la web como un grafo dirigido, donde cada nodo representa un sitio y los vertices van de un nodo a otro si un sitio contiene un link direccionando a este otro.
A partir de esto se define una \emph{matriz de vínculos} $M$, y como su nombre lo indica, esta representará las conectividades de los sitios pero con la particularidad de que tendrá información acerca de cuanto puntaje le asigna un sitio a otro, es decir, $m_{ij}$ será 1/$G_j$  (siendo $G_j$ el grado de salida del sitio $j$) si $j$ tiene un link a $i$ y 0 en otro caso.
De esta forma, la importancia de cada sitio esta medida por la calidad de sitios que linkean a cada uno, y a su vez la calidad esta conformada por la cantidad de sitios a los que redirijo, ya que si un sitio redirige a pocos sitios a cada uno estara otorgandole un puntaje cercano al 1, y en caso contrario el puntaje que otorgará sera cercano al 0 y no tendrá mucho peso a la hora de que los sitios de destino obtengan mayor importancia gracias a este.\\
Facilmente se puede deducir que todas las columnas de la matriz resultante van a sumar 1, por lo que será una matriz \textit{``estocastica por columnas"}. Luego el problema de encontrar el ranking de cada cada sitio es equivalente a encontrar un $x\in \mathbb{R}^n$ tal que $Px = x$, es decir, encontrar (suponiendo que existe) un autovector asociado al autovalor 1 de una matriz cuadrada, tal que $x_i \ge
0$ y $\sum_{i = 1}^n x_i = 1$. 
\\
Uno puede abstraerse a la noción del \emph{navegante aleatorio}. Este vendría a representar cualquier usuario común que navega por la web a traves de los links y que elige cualquiera de ellos con igual probabilidad.
\\
Un detalle importante que se debe considerar es que para que el PageRank funcione no puede haber nodos desconectados, ya que de ser así la matriz de vínculos no sería estocastica por columnas debido a que la columna de la matriz que representa a los sitios que linkea estarían todos en 0. La solución a esto vendría a ser la inversa del nodo, y es considerar a que el sitio esta conectado a \textbf{todos} los demás sitios, por lo que será equiprobable la navegación a cada uno de ellos. Para tal modificar se le debe hacer una modificación a la matriz original, con lo que si definimos $v \in \mathbb{R}^{n \times n}$, con $v_i = 1/n$ y $d \in \{0,1\}^{n}$ donde 
$d_i = 1$ si $G_i = 0$, y $d_i = 0$ en caso contrario, siendo $n$ la cantidad total de sitios web. La nueva matriz de transici\'on será:


\begin{eqnarray*}
D & = & v d^t \\
M_p & = & M + D.
\end{eqnarray*}
 

 Por esta razón es que Bryan y Leise \cite{Bryan2006} introducen la noción de \emph{teletransportación}. Este vendría a representar cualquier usuario común que navega por la web a traves de los links 


Una interpretaci\'on equivalente para el problema es considerar al \emph{navegante aleatorio}. \'Este empieza en una
p\'agina cualquiera del conjunto, y luego en cada p\'agina $j$ que visita sigue navegando a trav\'es de sus links,
eligiendo el mismo con probabilidad $1/n_j$. Una situaci\'on particular se da cuando la p\'agina no tiene links salientes. En
ese caso, consideramos que el navegante aleatorio pasa a cualquiera de las p\'agina de la red con probabilidad $1/n$.
Para representar esta situaci\'on, definimos $v \in \mathbb{R}^{n \times n}$, con $v_i = 1/n$ y $d \in \{0,1\}^{n}$ donde 
$d_i = 1$ si $n_i = 0$, y $d_i = 0$ en caso contrario. La nueva matriz de transici\'on es 
\begin{eqnarray*}
D & = & v d^t \\
P_1 & = & P + D.
\end{eqnarray*}
Adem\'as, consideraremos el caso de que el navegante aleatorio, dado que se encuentra en la p\'agina $j$, decida visitar
una p\'agina cualquiera del conjunto, independientemente de si esta se encuentra o no referenciada por $j$ (fen\'omeno
conocido como \emph{teletransportaci\'on}). Para ello, consideramos que esta decisi\'on se toma con una probabilidad
$c \ge 0$, y podemos incluirlo al modelo de la siguiente forma:
\begin{eqnarray*}
E & = & v \bar{1}^t \\
P_2 & = & cP_1 + (1-c)E,
\end{eqnarray*}
\noindent donde $\bar{1} \in \mathbb{R}^n$ es un vector tal que todas sus componenetes valen 1. La matriz resultante
$P_2$ corresponde a un enriquecimiento del modelo formulado en (\ref{eq:basicmodel}). Probabil\'isticamente, la
componente $x_j$ del vector soluci\'on (normalizado) del sistema $P_2 x = x$ representa la proporci\'on del tiempo que,
en el largo plazo, el navegante aleatorio pasa en la p\'agina $j \in Web$.

En particular, $P_2$ corresponde a una
matriz \emph{estoc\'astica por columnas} que cumple las hip\'otesis planteadas en Bryan y Leise \cite{Bryan2006} y
Kamvar et al. \cite{Kamvar2003}, tal que $P_2$ tiene un autovector asociado al autovalor 1, los dem\'as autovalores de
la matriz cumplen $1 = \lambda_1 > |\lambda_2| \ge \dots \ge |\lambda_n|$ y, adem\'as, la dimensi\'on
del autoespacio asociado al autovalor $\lambda_1$ es 1. Luego, la soluci\'on al sistema $P_2 x = x$ puede ser calculada
de forma est\'andar utilizando el m\'etodo de la potencia.

Una vez calculado el ranking, se retorna al usuario las $t$ p\'aginas con mayor ranking.


\subsubsection{HITS}

Este algoritmo fué pensado por Kleinberg $[2]$ y su esenscia está en la separación conceptual que hace entre un nodo \textit{autoridad} y uno \textit{Hub}. Vale aclarar que en este caso también entenderemos a un nodo como parte del grafo que modela a la internet y que representa a un sitio web en particular. \\
Autoridad serían aquellos sitios que tienen mayor importancia dentro de un tema específico y Hubs los que apuntan a estos sitios. En palabras mas terrenales las autoridades serían los que tienen las mejores respuestas y los hubs los que conocen a las mejores autoridades.
En la práctica se entiende por autoridades a aquellos nodos que son apuntados por una gran cantidad de sitios y por Hubs a los nodos a apuntan a muchos otros. Una web es mejor Autoridad si es apuntada por buenos Hubs y viceversa, un Hub es mejor si apunta a las mejores autoridades.\\
Para modelarlo computacionalmente pensamos a la red como una matriz $A \in \{0,1\}^{n \times n}$ donde cada fila y columna representan a un nodo y tienen un 1 si el nodo fila apunta al nodo columna o un 0 en caso contrario. Luego Kleinberg nos señala que debemos crear un vector \textit{x} e \textit{y} para agrupar las autoridades y hubs respectivamente. Estos vectores nos dice que los obtengamos tomando un x0 e y0 inciales con todos sus valores 1 y realizando este cálculo:

\begin{eqnarray}
x & = & A^ty \label{eq:auth-update-math} \\
y & = & Ax, \label{eq:hub-update-math} 
\end{eqnarray}

n veces (para un n definido) o hasta obtener un error menor a la tolerencia deseada. El error lo obtenemos calculando la norma manhattan entre el vector obtenido en el paso actual y el previo a este (mas adelante explicaremos como es la norma esta).\\
Luego de esto nos quedan en cada vector ordenados en el vector $x$ las mejores autoridades y en el $y$ los mejores hubs.

Dicho procedimiento debe aplicarse sobre una subred (denominada root-set) que debe calcularse previamente. Para esto mediante algún buscador simple, basado en texto por ejemplo, se acota la red a una determinada cantidad de nodos. Luego se le agregan aquellos que apuntan a algún nodo de dicha sub-red y los apuntados por esta, para finalmente sí, a este root-set aplicarle HITS. En este trabajo supondremos que este acotamiento previo ya fué efectuado.


\subsubsection{Indeg}
A modo de comparación para los experimentos, agregamos este algoritmo que resulta un poco inocente. Supongamos que tenemos una red de páginas, el peso, o importancia, de cada una será el promedio de la cantidad de links existentes en otras páginas del mismo conjunto hacia esta sobre la cantidad de links en total. El cálculo del mismo es lineal en la cantidad de referencias y su complejidad espacial es un vector de tamaño igual que el conjunto inicial de páginas. Este vector final es el resultado.

\begin{center}
$[k_1/n, k_2/n, ..., k_n/n]\quad 0 \leq k_i \leq n \quad n = cantTotalLinks$
\end{center}

\subsection{Norma Manhattan}

A lo largo de esta presentación, utilizaremos la distancia L1 entre dos vectores, también llamada Norma Manhattan.
La norma Manhattan es la suma de la diferencia coordenada a coordenada en modulo:

\begin{center}
$\left \| v - w \right \|_{1} = \sum_{i=1}^{n} \left | v_{i} - w_{i} \right |$
\end{center}

El uso del mismo estará detallado en el desarrollo de cada algoritmo presentado.


\subsection{Matriz Dispersa}
   Se define como matriz dispersa a aquella a la que la mayoría de sus elementos son cero.
   Ejemplo:

   $$ 
\begin{bmatrix}
       0    &      0    &   0       &   0           &   a_{04}    \\
       0    &   a_{11}  &   a_{12}  &   0           &   0    \\
       0    &      0    &   0       &   a_{23}      &   0    \\
       0    &      0    &   0       &   a_{33}      &   0    \\
  a_{40}    &      0    &   0       &   0           &   0    \\
\end{bmatrix} 
$$


\subsection{DOK vs CRS vs CSC}
    La matriz dispersa al tener la propiedad de tener muy pocos valores no$-$cero es conveniente solo guardar estos y asumir el resto como cero. Existen varias estructuras como Dictionary of Keys (dok), Compressed Sparse Row (CSR) o Compressed Sparse Column (CSC) pensadas para optimizar el espacio y las operaciones con estas estructuras de datos. En el desarrollo de este TP, utilizamos DOK por facilidad en el uso del mismo. Tanto CSR o CSC se basan en la estructura Yale y se diferencian en como guardan los mismos valores, uno priorizando las filas y otro las columnas respectivamente.\\
    La estructura Yale consiste en a partir de la matriz original obtener tres vectores que contengan 
    \begin{itemize}
        \item A = los elementos no$-$cero de arriba-abajo,izquierda-derecha
        \item IA = los indices para cada fila i del primer elemento no-cero de dicha fila
        \item JA = los indices de columna para cada valor de A
    \end{itemize}
    Si bien el caso de que haya una fila con muchos números no-ceros es más beneficiosa la utilización de esta estructura, la facilidad con DOK permite hacer pruebas más rápido.Y nos pareció poco práctico ponernos a implementar todas las lógicas requeridas para la eliminación o agregación de nuevos datos en estas estructuras ya que no hacían a la escencia del TP y complejizaban el código y el debagueo durante las pruebas y el desarrollo. Consideramos que la optimización otorgada por DOK es suficiente para el tipo de análisis que deseamos hacer sobre los algoritmos de rankeo solicitados


