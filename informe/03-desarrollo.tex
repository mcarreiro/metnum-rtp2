\section{Desarrollo}

A continuación detallamos cómo fue el desarrollo de los algoritmos presentados previamente

\subsection{Page Rank}
El algoritmo de PageRank lo dividimos en dos etapas, primero la inicializaci\'on en donde se crea la matriz estoc\'astica y luego la corrida en donde se itera y calcula el pagerank hasta que la diferencia de norma entre los vectores sea menor que la tolerancia establecida.

Inicializaci\'on:


\begin{algorithm}
\caption{initialize(c, dim, links)}\label{euclid}
\begin{algorithmic}[1]
\State $\textit{vectorInicial = vector(dim, 0);}$ \Comment{creo un vector de dim elementos y lo inicializo en 0}
\State $\textit{vectorInicial[0] = 1;}$ \Comment{pongo el primer elemento en 1}
\State $\textit{matriz = DOK(dim);}$ \Comment{la matriz representará al $M_f$}
\For{$cada\ link\ en\ links$}
\If{$link\ tiene\ salidas$}
\For{$cada\ salida\ del\ nodo$}
\State $\textit{;}$ \Comment{}
\EndFor
\Else
\State $\textit{desconectados.agregar(link);}$ \Comment{}
\EndIf
\EndFor


\end{algorithmic}
\end{algorithm}

Calculo del PageRank:
\begin{lstlisting}[frame=single] 
Hasta que converja:
	Multiplico la matriz por el vector actual.
	Aplico el algoritmo para tener en cuentas los nodos desconectados.
	Aplico el algoritmo para tener en cuenta el navegante aleatorio.
	Guardo el vector actual.
	
\end{lstlisting}


\subsection{HITS}
Este también lo dividimos en la etapa en la etapa de inicialización y de cálculo de sus vectores. En la primera creamos la matriz estocástica e inicializamos los vectores y en la segunda calculamos los mismos hasta que iteremos k veces o la diferencia obtenida sea menor que la tolerancia.

%Inicializaci\'on:
%\begin{lstlisting}[frame=single]  
%Creo el Dok vacio
%Para cada arista:
%	defino en el dok el nodo desde y hasta
%	
%Inicio los vectores de hubs y autoridades con todos sus valores en 1 y normalizados
	
%\end{lstlisting}

%Cálculo de vectores Hubs y Autoridades
%\begin{lstlisting}[frame=single] 
%Itero de 1 a K
%	Para el vector de hubs multiplico el dok transpuesto por el vector de autoridades y normalizo
%	Para el vector de autoridades multiplicoel dok transpuesto por el vector de hubs y normalizo
%	Si la diferencia entre el nuevo vector de hubs o autoridades con su valor previo es menor a la tolerancia termino la iteracion
%\end{lstlisting}
\subsection{Indeg}

La implementación de este algoritmo es bastante simple. Tomamos un vector inicial con ceros del tamaño de las páginas y recorremos todas las referencias de cada página hacia al resto, y por cada una de los sitios a los que visita, le sumamos en 1/cantidadLinksTotal su puntaje en el vector inicial.

\begin{lstlisting}[frame=single] 
Inicializo el vector de resultados con ceros.
Para cada conjunto de referencias de una pagina:
	Para cada referencia:
		Al vector resultados le sumo 1 / cantidad total de los links
	
\end{lstlisting}
