\section{Resumen}

Los sitios web a medida que fueron creciendo en cantidad en la época de los 90's, se complicó el acceso a ellos y ,a menos que alguien te comentara o a través de publicidades, era muy dificil acceder a la información deseada. Es por eso que se produjo el auge de los buscadores, que a partir de palabras claves, podrían devolverte sitios que  puedan llegar a responder tu pregunta o decirte algo al respecto. 
Un primer problema de entrada, es que, como todo en la vida, la calidad de dicho contenido puede no ser el deseado y existan mejores. Durante este trabajo repasaremos 3 algoritmos conocidos de ranqueo de páginas web, veremos los resultados y los compararemos. \\
Una vez que sepamos cómo funcionan y cómo ordenan y ubican los resultados, intentaremos responder a la pregunta: cuáles son los pasos a seguir para poder mejorar tu sitio y que salga con mejor puntaje que la competencia.\\
Durante el desarrollo del trabajo práctico comprobaremos que la forma de posicionarse en las búsquedas depende, en todos los algoritmos, de los links que otras páginas tengan a tu sitio (y viceversa), por lo tanto lo que concluiremos es que si el cliente quiere posicionarse bien, en el menor tiempo posible, habrá que recomendarle que negocie con otras páginas web para ser apuntadas por estas. La diferencia es que no necesariamente serán las mismas, esto dependerá de que método de búsqueda se esté utilizando.
