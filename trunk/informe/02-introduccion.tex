\section{Introducci\'on te\'orica}

\subsection{Matriz Dispersa}
   Se define como matriz dispersa a aquella a la que la mayoría de sus elementos son cero.
   Ejemplo:

   $$ 
\begin{bmatrix}
       0    &      0    &   0       &   0           &   a_{04}    \\
       0    &   a_{11}  &   a_{12}  &   0           &   0    \\
       0    &      0    &   0       &   a_{23}      &   0    \\
       0    &      0    &   0       &   a_{33}      &   0    \\
  a_{40}    &      0    &   0       &   0           &   0    \\
\end{bmatrix} 
$$

\subsection{DOK vs CRS vs CSC}
    La matriz dispersa al tener la propiedad de tener muy pocos valores no$-$cero es conveniente solo guardar estos y asumir el resto como cero. Existen varias estructuras como Dictionary of Keys (dok), Compressed Sparse Row (CSR) o Compressed Sparse Column (CSC) pensadas para optimizar el espacio y las operaciones con estas estructuras de datos. En el desarrollo de este TP, utilizamos DOK por facilidad en el uso del mismo. Tanto CSR o CSC se basan en la estructura Yale y se diferencian en como guardan los mismos valores, uno priorizando las columnas y otro las filas respectivamente.\\
    La estructura Yale consiste en a partir de la matriz original obtener tres vectores que contengan 
    \begin{itemize}
        \item A = los elementos no$-$cero de arriba-abajo,izquierda-derecha
        \item IA = los indices para cada fila i del primer elemento no-cero de dicha fila
        \item JA = los indices de columna para cada valor de A
    \end{itemize}
    Si bien en caso de que haya en una fila con muchos números no-ceros es más beneficioso la utilización de esta estructura, la facilidad con DOK permite hacer pruebas más rápido.Y nos pareció poco práctico ponernos a implementar todas las lógicas requeridas para la eliminación o agregación de nuevos datos en estas estructuras ya que no hacían a la escencia del TP y complejizaban el código y el debagueo durante las pruebas y el desarrollo. Consideramos que la optimización otorgada por DOK es suficiente para el tipo de análisis que deseamos hacer sobre los algoritmos de rankeo solicitados.